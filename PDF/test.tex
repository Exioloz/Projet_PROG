\documentclass[a4paper]{article} %type of paper

\usepackage[french]{babel} %paper language
\usepackage[utf8]{inputenc} %input language
\usepackage[T1]{fontenc} %fonts
\usepackage{amsmath} %insert advanced math
\usepackage{amssymb} %insert math symbols
\usepackage[hidelinks]{hyperref} %insert hyperlinks
\usepackage{graphicx} %insert images
\usepackage{listings} %insert code
\usepackage{soul} %underlines(ul)/barrer(st)
\usepackage{xcolor} %use new colors

% color for code
\definecolor{backColor}{HTML}{fdfdfd}
\definecolor{commentsColor}{HTML}{008000}
\definecolor{keywordsColor}{HTML}{1e7fe8}
\definecolor{numberColor}{HTML}{898989}
\definecolor{stringColor}{HTML}{bd4bb0}

% title
\title{PROG 5 Projet 2021-2022 \\
\large Documentation des Testes Effectués}
% Author
\author{Kimberly Beauvais, Xuan Li, Nathaniel Tobing \\
Hugo Roger, Emilien Maillard-Simon, Théo Lanneau}


\lstdefinestyle{mystyle}{
    language=C,
    frame=single,
    backgroundcolor=\color{backColor},   
    commentstyle=\color{commentsColor},
    keywordstyle=\color{keywordsColor}\bfseries,
    numberstyle=\tiny\color{numberColor},
    stringstyle=\color{stringColor},
    basicstyle=\ttfamily\footnotesize,
    breakatwhitespace=false,         
    breaklines=true,                 
    captionpos=b,                    
    keepspaces=true,                 
    numbers=left,                    
    numbersep=10pt,                  
    showspaces=false,                
    showstringspaces=false,
    showtabs=false,                  
    tabsize=2
}

\lstset{style=mystyle}

\begin{document}

% title 
\maketitle
\begin{center}
\rule{\textwidth}{1pt}
\end{center}

% table of contents
\renewcommand{\contentsname}{Table des Matières}
\tableofcontents

% Phase 1
\section{Phase 1}
Le but des tests est d’exécuter avec différentes options la fonction créée 
Readelf sur un jeu significatif de programmes ARM compilés, puis de comparer 
les résultats à ceux obtenus par l’exécution de la commande Readelf existante.


%1.1
\subsection{Affichage de l’en-tête}
On teste d’abord l’affichage de l’entête des fichiers ELF avec un script qui 
nettoie les sorties des deux fonctions puis les compare. Si les 2 sorties sont 
identiques, le test est validé.

%1.2
\subsection{Affichage de la table des sections}
On teste ensuite l’affichage de l’entête des sections des fichiers ELF.

%1.3
\subsection{Affichage du contenu d'une section}
Ici on veut tester l’affichage des hex-dumps des différentes sections des 
fichiers ELF. Pour chaque fichier ELF on va chercher l’ensemble des noms de 
ses différentes sections via un nettoyage de la sortie de la fonction Readelf -h.
Puis pour chaque section on compare les hex-dumps des deux fonctions.
Si une section est vide elle est filtrée et donc pas testée.


%1.4
\subsection{Affichage de la table des symboles}
On compare ensuite les deux tables des symboles obtenues à partir des deux 
fonctions.

%1.5
\subsection{Affichage des tables de réimplantation}
Finalement on compare les deux tables de réadressage obtenues à partir des 
deux fonctions.
% Phase 2
\section{Phase 2}
Dans cette partie, les tests sont faits manuellement. Il s’agit de comparer 
la sortie faite par la commande 

\begin{lstlisting}
    ./relocation .text=<nombre> .data=<nombre> 
      Examples_loader/<fichier.o>
\end{lstlisting}
et celle avec l’utilisation de la commande 
\begin{lstlisting}
    readelf -hSs Examples_loader/<fichier.o>
\end{lstlisting}


%2.6
\subsection{Rénumerotation des sections}
Il faut vérifier que les sections du type REL ont bien été supprimées et les 
sections sont bien renumérotées et aussi que l’adresse des sections .text et 
.data sont changées.

%2.7
\subsection{Correction des symboles}
Il faut verifier que les valeurs et les ndx des symboles dans la table des symboles 
ont bien été changés et que les ndx pointe vers la même section qu’avant

%2.8
\subsection{Réimplantation de type ARM}
Il faut lire le contenu des sections affectées par la réimplantation (readelf -x) 
et vérifier que les valeurs sont bien changées.(pour vérifier l’écriture du nouveau 
fichier binaire) Il faut faire 
\begin{lstlisting}
    readelf -hSs out.bin 
\end{lstlisting}
et vérifier que l’affichage est le même que la sortie de la commande ./relocation

\section{Différent Testes}
\subsection{Tests Automatique}

\noindent Ces tests sont effectués automatiquement en lançant le scripte 
\begin{lstlisting}
    ./tests\_part1.sh
\end{lstlisting}

\noindent \textbf{example5.s} :
Ce test cherche à vérifier si le programme fonctionne en chargeant différentes 
données dans le même registre. 
\bigbreak

\noindent \textbf{example6.s} :
Ce test cherche à vérifier si le programme fonctionne en ignorant une liste 
d’instructions
\bigbreak

\noindent \textbf{example7.s} :
Ce test cherche à vérifier si le programme fonctionne en n’ayant aucune 
instruction
\bigbreak

\noindent \textbf{example8.s} : 
Ce test cherche à vérifier si le programme fonctionne en ayant un .text vide et 
un .data rempli
\bigbreak

\noindent \textbf{example9.s} :
Ce test cherche à vérifier si le programme fonctionne en ayant plusieurs balises 
inutilisées
\bigbreak

\noindent \textbf{example10.s} :
Ce test cherche à vérifier si le programme fonctionne en utilisant une pile
\bigbreak

\noindent \textbf{example11.s} :
Ce test cherche à vérifier si le programme fonctionne en ayant une boucle avec des 
add et mov
\bigbreak

\noindent \textbf{example12.s} :
Ce test cherche à vérifier si le programme fonctionne en ayant plusieurs données de 
différents types
\bigbreak

\noindent \textbf{example13.s} :
Ce test cherche à vérifier si le programme fonctionne en utilisant des opérations 
logiques
\bigbreak

\noindent \textbf{example14.s} :
Ce test cherche à vérifier si le programme fonctionne en utilisant les différents 
flags pour des branchements
\bigbreak

\noindent \textbf{example15.s} :
Ce test cherche à vérifier si le programme fonctionne en ajoutant plusieurs 
directives différentes
\bigbreak

\subsection{Tests Manuels}
 
\noindent Tests manuels pour la partie 2

\noindent Test simple
\begin{lstlisting}
    ./relocation .text=0x20 .data=0x2800 Examples\_loader/example1.o
\end{lstlisting}
\bigbreak

\noindent Test avec adresses nulles (donc aucun changement) :
\begin{lstlisting}
    ./relocation .text=0 .data=0 Examples\_loader/example1.o
\end{lstlisting}
\bigbreak

\noindent Test avec un programme sans section .data dans le fichier .s d'origine 
(aucune différence, car la section .data est quand même créée dans le .o lors 
de la compilation)
\begin{lstlisting}
    ./relocation .text=0x20 .data=0x2800 Examples\_loader/example2.o
\end{lstlisting}
\bigbreak

\noindent Tests avec adresses données en argument étant des nombres négatifs:
Avec ce programme, on interdit les adresses négatives. Au niveau du code, 
normalement cela fonctionne mais les numéros d'adresses deviennent très grands 
car on a implémenté avec un type unsigned.
\bigbreak
\noindent - test avec section .text a une nouvelle adresse étant un nombre négatif
\begin{lstlisting}
    ./relocation .text=-0x34 .data=0x2800 Examples\_loader/example1.o
\end{lstlisting}

\noindent - test avec section .data a une nouvelle adresse étant un nombre négatif
\begin{lstlisting}
    ./relocation .text=0 .data=-0x11 Examples\_loader/example1.o
\end{lstlisting}


\noindent - test avec fichier exécutable (donne un message d’erreur, seuls les fichiers 
avec sections de relocation sont permis) :
\begin{lstlisting}
    ./relocation .text=0 .data=-0x11 Examples\_loader/example10
\end{lstlisting}



\end{document}