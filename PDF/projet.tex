\documentclass[a4paper]{article} %type of paper

\usepackage[french]{babel} %paper language
\usepackage[utf8]{inputenc} %input language
\usepackage[T1]{fontenc} %fonts
\usepackage{soul} %underlines(ul)/barrer(st)
\usepackage{listings} %insert code
\usepackage{xcolor} %use new colors
\usepackage{graphicx} %insert images
\usepackage{amsmath} %insert advanced math
\usepackage{amssymb} %insert math symbols
\usepackage[hidelinks]{hyperref} %insert hyperlinks

\definecolor{backColor}{HTML}{fdfdfd}
\definecolor{commentsColor}{HTML}{008000}
\definecolor{keywordsColor}{HTML}{1e7fe8}
\definecolor{numberColor}{HTML}{898989}
\definecolor{stringColor}{HTML}{bd4bb0}

\usepackage{amsmath}
\title{PROG 5 Projet 2021-2022}
\author{Kimberly Beauvais, Xuan Li, Hugo Roger, Akira Tobing}


\lstdefinestyle{mystyle}{
    language=C,
    frame=single,
    backgroundcolor=\color{backColor},   
    commentstyle=\color{commentsColor},
    keywordstyle=\color{keywordsColor}\bfseries,
    numberstyle=\tiny\color{numberColor},
    stringstyle=\color{stringColor},
    basicstyle=\ttfamily\footnotesize,
    breakatwhitespace=false,         
    breaklines=true,                 
    captionpos=b,                    
    keepspaces=true,                 
    numbers=left,                    
    numbersep=10pt,                  
    showspaces=false,                
    showstringspaces=false,
    showtabs=false,                  
    tabsize=2
}

\lstset{style=mystyle}

\begin{document}

\maketitle
\begin{center}
\rule{\textwidth}{1pt}
\end{center}


\renewcommand{\contentsname}{Table des Matières}
\tableofcontents

% -------------------------------------------------------------------
%                    Principe du Projet 
% -------------------------------------------------------------------
\section*{Principe du Projet}

TODO!!!

% -------------------------------------------------------------------
%                    Phase 1 - Readelf
% -------------------------------------------------------------------

\section{Phase 1 : Lecture du Format ELF }
\label{sec:phase1}
\noindent La première phase du projet consiste à lire un fichier elf et produire une sortie similaire à readelf. \textit{./readelf} affiche des informations sur un objet au format ELF.  Les options contrôlent les informations particulières à afficher. \textit{elffile}... sont les fichiers objets à examiner.  Les fichiers ELF 32 bits sont supportés, tout comme les archives contenant des fichiers ELF sont supportés. Ce programme exécute une fonction similaire à \textit{objdump} et \textit{readelf} mais avec seulement les fonctionnalités spécifiées dans la sous-section \hyperref[sec:options]{options}. 


% Readelf - main function

\subsection{readelf}
\label{sec:readelf}
Ce fichier contient le programme principal de la phase 1 de ce projet. 

% elf_main - options and process file data

\subsection{elf\_main}

\subsubsection*{Big Endian}

\subsubsection*{Options}
\label{sec:options}

% elf_header - get and process file header

\subsection{elf\_header}
\label{sec:elfheader}
Ce fichier contient les fonctions permettant d'afficher les informations contenues dans l'en-tête ELF au début du fichier.

\subsubsection*{Obtenir l'en-tête du fichier}

\subsubsection*{Traitement de l'en-tête du fichier}

% elf_shdrs - get and process section header table

\subsection{elf\_shdrs}
\label{sec:sectionheader}
Ce fichier contient les fonctions permettant d'afficher les informations contenues dans les en-têtes de section du fichier, si celui-ci en possède.

\subsubsection*{Obtenir les en-têtes des sections}

\subsubsection*{Traitement de les en-têtes des sections}

% read_section - get and process section

\subsection{read\_section}
\label{sec:section}

TO DO!!!!

% elf_symbol_table - get and process symbol table

\subsection{elf\_symbol\_table}
\label{sec:symbol}

TO DO!!!!

% elf_reltab - get and process relocation table

\subsection{elf\_reltab}
\label{sec:relocation}

TO DO!!!!

% -------------------------------------------------------------------
%                    Phase 2 Reimplementation 
% -------------------------------------------------------------------

\section{Phase 2}
\label{sec:phase2}
TO DO!!!!

\subsection{Renumérotation des sections}
TO DO!!!!

\subsection{Correction des symboles}
TO DO!!!!

\subsection{Réimplémentations de type R\_ARM\_ABS*}
TO DO!!!!

\subsection{Réimplémentation de type R\_ARM\_JUMP24 et R\_ARM\_CALL}
TO DO!!!!

\subsection{Interfaçage avec le simulateur ARM}
TO DO!!!!

\subsection{Production d’un fichier exécutable non relogeable}

TO DO!!!!

\section{Bugs Pas Resolu}
\label{sec:bug}

TO DO!!!! (hopefully none lol)


\section{Testes}
\label{sec:test}

\end{document}