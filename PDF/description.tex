\documentclass[a4paper]{article} %type of paper

\usepackage[french]{babel} %paper language
\usepackage[utf8]{inputenc} %input language
\usepackage[T1]{fontenc} %fonts
\usepackage{amsmath} %insert advanced math
\usepackage{amssymb} %insert math symbols
\usepackage{enumitem} %lists 
\usepackage[hidelinks]{hyperref} %insert hyperlinks
\usepackage{graphicx} %insert images
\usepackage{listings} %insert code
\usepackage{soul} %underlines(ul)/barrer(st)
\usepackage{tabto}
\usepackage{xcolor} %use new colors

% color for code
\definecolor{backColor}{HTML}{fdfdfd}
\definecolor{commentsColor}{HTML}{008000}
\definecolor{keywordsColor}{HTML}{1e7fe8}
\definecolor{numberColor}{HTML}{898989}
\definecolor{stringColor}{HTML}{bd4bb0}

% title
\title{PROG 5 Projet 2021-2022 \\
\large Descriptif de la Structure du Code Dévéloppé}
% Author
\author{Kimberly Beauvais, Xuan Li, Nathaniel Tobing \\
Hugo Roger, Emilien Maillard-Simon, Théo Lanneau}


\lstdefinestyle{mystyle}{
    language=C,
    frame=single,
    backgroundcolor=\color{backColor},   
    commentstyle=\color{commentsColor},
    keywordstyle=\color{keywordsColor}\bfseries,
    numberstyle=\tiny\color{numberColor},
    stringstyle=\color{stringColor},
    basicstyle=\ttfamily\footnotesize,
    breakatwhitespace=false,         
    breaklines=true,                 
    captionpos=b,                    
    keepspaces=true,                 
    numbers=left,                    
    numbersep=10pt,                  
    showspaces=false,                
    showstringspaces=false,
    showtabs=false,                  
    tabsize=2
}

\lstset{style=mystyle}

\begin{document}

% title 
\maketitle
\begin{center}
\rule{\textwidth}{1pt}
\end{center}

% table of contents
\renewcommand{\contentsname}{Table des Matières}
\tableofcontents

\newpage

%---------------------------------------------------------------------
% Principe du Projet 
%---------------------------------------------------------------------

\section*{Principe du Projet}
L'objectif de ce projet est d’implémenter une sous partie d’un éditeur de liens. 
Plus précisément, le projet est centré sur la dernière phase, dite de réimplantation, 
exécut´ee par l’éditeur de liens. Ce document contient une description générale du 
code créé tout au long du projet, qui est divisé en deux phases : 
\hyperref[sec:phase1]{Fusion} et \hyperref[sec:phase2]{Implantation}

%---------------------------------------------------------------------
% Phase 1 - readelf
%---------------------------------------------------------------------

\section{Phase 1 : Fusion }
\label{sec:phase1}
La première phase du projet consiste à rassembler les différentes zones (sections)
définis dans les fichiers objets donnés en entrée. Le programme principal, \hyperref[sec:readelf]{\textit{./readelf}}, 
affiche des informations sur un objet au format ELF.  Les options contrôlent les 
informations particulières à afficher. \textit{elffile} \label{sec:elffile}... sont les fichiers 
objets à examiner.  Les fichiers ELF 32 bits sont supportés, tout comme les 
archives contenant des fichiers ELF sont supportés. Ce programme exécute une 
fonction similaire à \textit{objdump} et \textit{readelf} mais avec seulement 
les fonctionnalités spécifiées dans la sous-section \hyperref[sec:options]{options}. 


Chaque étape est divisée en deux parties, la première qui récupère les données 
du fichier et la seconde qui traite les données et imprime le résultat.

% elf_header - get and process file header

\subsection{elf\_header}
\label{sec:elfheader}
\noindent Ce fichier contient les fonctions permettant d'afficher les informations 
contenues dans l'en-tête ELF au début du fichier.

\subsubsection*{Obtenir l'en-tête du fichier}
\label{sec:getfheader}
Afin d'obtenir l'en-tête du fichier, on lis les 16 premiers bytes dans \hyperref[sec:elffile]{\textit{elffile}}
et le stocker dans le structure : \textit{filedata->header.e\_ident}.
Après avoir vérifié que le fichier est bien un fichier ELF, on lis le 
reste de l'en-tête du fichier, effectuons l'opération big endian sur chaque 
structure et le stocker dans \hyperref[sec:filedata]{\textit{filedata->header}}.

\subsubsection*{Traitement de l'en-tête du fichier}
Pour afficher les données dans \hyperref[sec:filedata]{\textit{filedata->header}}, 
on a crée des sous fonctions pour obtenir le bon nom selon les valeurs obtenu depuis 
\hyperref[sec:getfheader]{\textit{get\_file\_header}}.



% elf_shdrs - get and process section header table

\subsection{elf\_shdrs}
\label{sec:sectionheader}
\noindent Ce fichier contient les fonctions permettant d'afficher les informations 
contenues dans les en-têtes de section du fichier, si celui-ci en possède.

\subsubsection*{Obtenir les en-têtes des sections}
Afin d'obtenir les en-têtes des sections, on verifie qu'il existe des en-têtes des 
sections depuis les données dans \textit{filedata->header}. S'il y en a, on
récupére la table d'en-tête de section à partir du \hyperref[sec:elffile]{\textit{elffile}} 
et pour chaque section, on effectue le fonction \hyperref[sec:endian]{big\_endian} et stocker 
le resultat dans \textit{filedata->section\_headers}. Dans cette fonction, on obtient 
également les données pour le tableau des chaînes de caractères pour obtenir le nom 
des sections.


\subsubsection*{Traitement de les en-têtes des sections}
Pour traiter les données dans \textit{filedata->section\_headers}, on a crée des sous 
fonctions pour obtenir le bon nom selon les valeurs obtenu depuis 
\textit{get\_section\_headers}. Ensuite, on affiche le clé des flags. 


% read_section - get and process section

\subsection{read\_section}
\label{sec:section}
\noindent Ce fichier contient les fonctions permettant d'afficher le section saisie s'il existe
dans \hyperref[sec:elffile]{\textit{elffile}}. 
\newline

La fonction \textit{read\_section} permet de lire le contenu brut 
(sous forme héxadécimale) d'une section saisie par l'utilisateur. Elle trouve 
l'emplacement de la section correspondante en comparant son nom avec ceux qui 
sont dans la section header et après, affiche le contenu dans terminal.


% elf_symbol_table - get and process symbol table

\subsection{elf\_symbol\_table}
\label{sec:symbol}
\noindent Ce fichier contient les fonctions permettant d'afficher le table des symboles.

\subsubsection*{Obtenir les en-têtes des sections}
\label{sec:getsymtab}
Afin d'obtenir le table de symboles, on récupère le string table depuis \hyperref[sec:elffile]{\textit{elffile}}. 
Ensuite, utilisant \textit{section\_headers[index\_symtab].sh\_offset} dans 
l'en tête des sections de la table de symboles, on récupère toutes les entrées.

\subsubsection*{Traitement de les en-têtes des sections}
Pour traiter les données dans \textit{filedata->symbol\_table}, on a crée des sous 
fonctions pour obtenir le bon nom ou valeur selon les données obtenu depuis 
\hyperref[sec:getsymtab]{\textit{get\_symbol\_table}}.


% elf_reltab - get and process relocation table

\subsection{elf\_reltab}
\label{sec:relocation}
\noindent Ce fichier contient les fonctions permettant d'afficher les tables de 
reimplementation.

\subsubsection*{Obtenir les tables de reimplementation}
\label{sec:getreltab}
Afin d'obtenir les tables de reimplementation du fichier, on commence par compter 
le nombre de sections de relocalisation qui se trouvent dans \hyperref[sec:elffile]{\textit{elffile}} pour alluer 
le bon taille de memoire à \textit{filedata->reloc\_table.rel\_tab}. Ensuite, si une 
section a le type REL, alors on ajoute les entrées de cette section dans le tableau de 
reimplantation.

\subsubsection*{Traitement de les tables de reimplementation}
Pour traiter les données dans \textit{filedata->reloc\_table}, on transform 
les bits des données obtenu depuis \hyperref[sec:getreltab]{\textit{get\_rel\_table}}
utilisant un fonction \hyperref[sec:endian]{big endian}. 

% elf_main - get and process file data

\subsection{elf\_main}
\noindent Ce fichier contient les fonctions pour obtenir le data et les processer 
selon les \hyperref[sec:options]{options} donnée pour l'utilisateur. Le but de ce 
fichier est de rassembler toutes les fonctions de la phase 1 pour obtenir un fichier 
de sortie homogène mais aussi d'obtenir un structure data utilisable pour la 
\hyperref[sec:phase2]{phase 2} 

\subsubsection*{Structures de Données}
\label{sec:filedata}

On a créé une nouvelle structure de données \textit{filedata} qui contient 
toutes les données nécessaires du fichier qui seront manipulées dans les \hyperref[sec:phase1]{phases 1} 
et \hyperref[sec:phase2]{2}. 

\begin{lstlisting}
typedef struct filedata {
    const char *        file_name;          // file name
    Elf32_Ehdr          file_header;        // file header
    FILE *              file;               // pointer to file
    uint32_t            file_offset;        // offset of file
    uint32_t            file_size;          // size of file
    Elf32_Shdr *        section_headers;    // section header table
    Elf32_Rel_Tab       reloc_table;        // relocation table
    Elf32_Sym_Tab       symbol_table;       // symbol table
    char *              string_table;       // string table
    int                 string_table_length;// string table size
} Filedata;
\end{lstlisting}

\subsubsection*{Big Endian}
\label{sec:endian}
Les fonctions big endians sont utilisé pour transoformer les bytes obtenu depuis 
le fichier objet en big endian. \textit{field} est le bit qu'on veut transformer. 
Pour n'importe quel type de donnée, le concept reste le même :
\begin{lstlisting}
    1 bit  : return field 
    2 bits : return (field[1] | field[0] << 8)
    3 bits : return (field[2] | field[1] << 8 | field[0] << 16);
    4 bits : return (field[3] | field[2] << 8 | field[1] << 16 | 
                     field[0] << 24);
\end{lstlisting}

\subsubsection*{Options}
\label{sec:options}
Dans le programme principal pour la phase 1, on a ajouté des options d'affichage 
selon les besoins d'utilisiateur. Les options sont les suivant:

\tabto{3em} \textbf{-a} \tabto{10em} Equivalent à: -h -S -s
\tabto{3em} \textbf{-e} \tabto{10em} Equivalent à: -h -S
\tabto{3em} \textbf{-h} \tabto{10em} Affichage de l'en-tête de fichier ELF
\tabto{3em} \textbf{-S} \tabto{10em} Affichage de la table des sections
\tabto{3em} \textbf{-s} \tabto{10em} Affichage de la table des symboles
\tabto{3em} \textbf{-x} \tabto{10em} Affichage du contenu d'une section
\tabto{3em} \textbf{-r} \tabto{10em} Affichage des tables de réimplantation

\subsubsection*{Obtenir les Données du fichier ELF}
\label{sec:data}
Le but de la fonction \textit{get\_filedata} est d'obtenir tout le données du 
structure \hyperref[sec:filedata]{\textit{filedata}} depuis la fichier d'objet donnée. Ce fonction va être 
utilisé aussi pour \hyperref[sec:phase2]{phase 2}. 

% readelf - main function

\subsection{readelf}
\label{sec:readelf}
Ce fichier contient le programme principal de la \hyperref[sec:phase1]{phase 1} de ce projet. 
Il était créer pour séparer les deux phases tout en réutilisant les fonctions créées en 
\hyperref[sec:phase1]{phase 1} dans la \hyperref[sec:phase2]{phase 2}.

%---------------------------------------------------------------------
% Phase 2 - reimplementation
%---------------------------------------------------------------------

\section{Phase 2}
\label{sec:phase2}
La deuxiéme phase du projet consiste à modifier le contenu du \hyperref[sec:elffile]{\textit{elffile}}
afin d'effectuer l'implantation. Le programme principal prend en paramètre 
les adresses auxquelles les sections du programme (typiquement .text et .data) 
doivent être chargées. Le résultat du programme est mis dans fichier nommé 
\hyperref[sec:bin]{\textit{out.bin}}. 

% relocation - main function

\subsection{relocation}

\subsubsection*{Renumérotation des sections}
TO DO!!!!

\subsubsection*{Correction des symboles}
TO DO!!!!

% reimplantation type
\subsection{reimplantation\_type}

\subsubsection*{Réimplantations de type R\_ARM\_ABS*}
TO DO!!!!

\subsubsection*{Réimplantation de type R\_ARM\_JUMP24 et R\_ARM\_CALL}
TO DO!!!!

\subsection{out.bin}
\label{sec:bin}
TO DO!!!!





\end{document}