\documentclass[a4paper]{article} %type of paper

\usepackage[french]{babel} %paper language
\usepackage[utf8]{inputenc} %input language
\usepackage[T1]{fontenc} %fonts
\usepackage{amsmath} %insert advanced math
\usepackage{amssymb} %insert math symbols
\usepackage{enumitem} %lists 
\usepackage[hidelinks]{hyperref} %insert hyperlinks
\usepackage{graphicx} %insert images
\usepackage{listings} %insert code
\usepackage{soul} %underlines(ul)/barrer(st)
\usepackage{tabto}
\usepackage{xcolor} %use new colors

% color for code
\definecolor{backColor}{HTML}{fdfdfd}
\definecolor{commentsColor}{HTML}{008000}
\definecolor{keywordsColor}{HTML}{1e7fe8}
\definecolor{numberColor}{HTML}{898989}
\definecolor{stringColor}{HTML}{bd4bb0}

% title
\title{PROG 5 Projet 2021-2022 \\
\large Descriptif de la Structure du Code Dévéloppé}
% Author
\author{Kimberly Beauvais, Xuan Li, Akira Tobing \\
Hugo Roger}


\lstdefinestyle{mystyle}{
    language=C,
    frame=single,
    backgroundcolor=\color{backColor},   
    commentstyle=\color{commentsColor},
    keywordstyle=\color{keywordsColor}\bfseries,
    numberstyle=\tiny\color{numberColor},
    stringstyle=\color{stringColor},
    basicstyle=\ttfamily\footnotesize,
    breakatwhitespace=false,         
    breaklines=true,                 
    captionpos=b,                    
    keepspaces=true,                 
    numbers=left,                    
    numbersep=10pt,                  
    showspaces=false,                
    showstringspaces=false,
    showtabs=false,                  
    tabsize=2
}

\lstset{style=mystyle}

\begin{document}

% title 
\maketitle
\begin{center}
\rule{\textwidth}{1pt}
\end{center}

% table of contents
\renewcommand{\contentsname}{Table des Matières}
\tableofcontents

\newpage

%---------------------------------------------------------------------
% Principe du Projet 
%---------------------------------------------------------------------

\section*{Principe du Projet}

TODO!!!

%---------------------------------------------------------------------
% Phase 1 - readelf
%---------------------------------------------------------------------

\section{Phase 1 : Fusion }
\label{sec:phase1}
La première phase du projet consiste à rassembler les différentes zones (sections)
définis dans les fichiers objets donnés en entrée. Le programme principal, \textit{./readelf}, 
affiche des informations sur un objet au format ELF.  Les options contrôlent les 
informations particulières à afficher. \textit{elffile}... sont les fichiers 
objets à examiner.  Les fichiers ELF 32 bits sont supportés, tout comme les 
archives contenant des fichiers ELF sont supportés. Ce programme exécute une 
fonction similaire à \textit{objdump} et \textit{readelf} mais avec seulement 
les fonctionnalités spécifiées dans la sous-section \hyperref[sec:options]{options}. 


Chaque étape est divisée en deux parties, la première qui récupère les données 
du fichier et la seconde qui traite les données et imprime le résultat.

% elf_header - get and process file header

\subsection{elf\_header}
\label{sec:elfheader}
\noindent Ce fichier contient les fonctions permettant d'afficher les informations 
contenues dans l'en-tête ELF au début du fichier.

\subsubsection*{Obtenir l'en-tête du fichier}
Afin d'obtenir l'en-tête du fichier, on lis les 16 premiers bytes dans le fichier 
objet donnée et le stocker dans le structure : \textit{filedata->header.e\_ident}.
Après avoir vérifié que le fichier est bien un fichier ELF, on lis le 
reste de l'en-tête du fichier, effectuons l'opération big endian sur chaque 
structure et le stocker dans \textit{filedata->header}.

\subsubsection*{Traitement de l'en-tête du fichier}
Pour traiter les données dans \textit{filedata->header}, 

% elf_shdrs - get and process section header table

\subsection{elf\_shdrs}
\label{sec:sectionheader}
\noindent Ce fichier contient les fonctions permettant d'afficher les informations contenues dans les en-têtes de section du fichier, si celui-ci en possède.

\subsubsection*{Obtenir les en-têtes des sections}

\subsubsection*{Traitement de les en-têtes des sections}

% read_section - get and process section

\subsection{read\_section}
\label{sec:section}
Ce fichier contient les fonctions permettant d'afficher le section saisie s'il existe
dans le fichier objet. 
\newline

La fonction \textit{read\_section} permet de lire le contenu brut 
(sous forme héxadécimale) d'une section saisie par l'utilisateur. Elle trouve 
l'emplacement de la section correspondante en comparant son nom avec ceux qui 
sont dans la section header et après, affiche le contenu dans terminal.


% elf_symbol_table - get and process symbol table

\subsection{elf\_symbol\_table}
\label{sec:symbol}

Le fichier elf\_symbol\_table contient plusieurs fonctions. Elle a deux 
fonctionnalités principales. L'une est de récupérer les données utiles 
(get\_symbol\_table) pour la suite, l'autre est d'afficher la table de symbole 
dans terminal (process\_symbol\_table).


% elf_reltab - get and process relocation table

\subsection{elf\_reltab}
\label{sec:relocation}

TO DO!!!!

% elf_main - get and process file data

\subsection{elf\_main}
Ce fichier contient les fonctions pour obtenir le data et les processer selon les 
\hyperref[sec:options]{options} donnée pour l'utilisateur. Le but de ce fichier est 
de rassembler toutes les fonctions de la phase 1 pour obtenir un fichier de sortie 
homogène mais aussi d'obtenir un structure data utilisable pour la 
\hyperref[sec:phase2]{phase 2} 

\subsubsection*{Structures de Données}
\label{sec:filedata}

Nous avons créé une nouvelle structure de données \textit{filedata} qui contient 
toutes les données nécessaires du fichier qui seront manipulées dans les phases 
1 et 2. 

\begin{lstlisting}
typedef struct filedata {
    const char *        file_name;          // file name
    Elf32_Ehdr          file_header;        // file header
    FILE *              file;               // pointer to file
    uint32_t            file_offset;        // offset of file
    uint32_t            file_size;          // size of file
    Elf32_Shdr *        section_headers;    // section header table
    Elf32_Rel_Tab       reloc_table;        // relocation table
    Elf32_Sym_Tab       symbol_table;       // symbol table
    char *              string_table;       // string table
    int                 string_table_length;// string table size
} Filedata;
\end{lstlisting}

\subsubsection*{Big Endian}
\label{sec:endian}
Les fonctions big endians sont utilisé pour transoformer les bytes obtenu depuis 
le fichier objet en big endian. \textit{field} est le bit qu'on veut transformer. 
Pour n'importe quel type de donnée, le concept reste le même :
\begin{lstlisting}
    1 bit  : return field 
    2 bits : return (field[1] | field[0] << 8)
    3 bits : return (field[2] | field[1] << 8 | field[0] << 16);
    4 bits : return (field[3] | field[2] << 8 | field[1] << 16 | 
                     field[0] << 24);
\end{lstlisting}

\subsubsection*{Options}
\label{sec:options}
Dans le programme principal pour la phase 1, on a ajouté des options d'affichage 
selon les besoins d'utilisiateur. Les options sont les suivant:

\tabto{3em} \textbf{-a} \tabto{10em} Equivalent à: -h -S -s
\tabto{3em} \textbf{-e} \tabto{10em} Equivalent à: -h -S
\tabto{3em} \textbf{-h} \tabto{10em} Affichage de l'en-tête de fichier ELF
\tabto{3em} \textbf{-S} \tabto{10em} Affichage de la table des sections
\tabto{3em} \textbf{-s} \tabto{10em} Affichage de la table des symboles
\tabto{3em} \textbf{-x} \tabto{10em} Affichage du contenu d'une section
\tabto{3em} \textbf{-r} \tabto{10em} Affichage des tables de réimplantation

\subsubsection*{Obtenir les Données}
\label{sec:data}
Le but de la fonction \hyperref[sec:filedata]{\textit{get\_filedata}} est d'obtenir tout le données du 
structure \textit{filedata} depuis la fichier d'objet donnée. Ce fonction va être 
utilisé aussi pour \hyperref[sec:phase2]{phase 2}. 

% readelf - main function

\subsection{readelf}
\label{sec:readelf}
Ce fichier contient le programme principal de la phase 1 de ce projet. 


%---------------------------------------------------------------------
% Phase 2 - reimplementation
%---------------------------------------------------------------------

\section{Phase 2}
\label{sec:phase2}
TO DO!!!!

\subsection{Renumérotation des sections}
TO DO!!!!

\subsection{Correction des symboles}
TO DO!!!!

\subsection{Réimplantations de type R\_ARM\_ABS*}
TO DO!!!!

\subsection{Réimplantation de type R\_ARM\_JUMP24 et R\_ARM\_CALL}
TO DO!!!!

\subsection{Interfaçage avec le simulateur ARM}
TO DO!!!!

\subsection{Production d’un fichier exécutable non relogeable}

TO DO!!!!



\end{document}