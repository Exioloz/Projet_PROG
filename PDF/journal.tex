\documentclass[11pt,letterpaper]{article}
\usepackage{pifont}
\usepackage{enumitem,amssymb}
\newlist{todolist}{itemize}{2}
\setlist[todolist]{label=$\square$}

% title
\title{PROG 5 Projet 2021-2022 \\
\large Journal - Progression du Travail et Repartition des Tâches}
% Author
\author{Kimberly Beauvais, Xuan Li, Nathaniel Tobing \\
Hugo Roger, Emilien Maillard-Simon, Théo Lanneau}

\date{}
\begin{document}
\maketitle

\noindent\rule{13cm}{0.4pt}

% Day 1 
\section*{Day 1 - 16/12/2021}

Après la présentation du projet, nous avons créé un serveur discord pour organiser 
la communication entre les membres du groupe. Le dépôt git a été créé et les membres 
ont été ajoutés. Trois documents Google ont également été créés :

\indent - Description du code, bogues, et testes

\indent - Journal pour le progression du travail et repartition des tâches

\indent - Document pour mettre des idées

\subsection*{Tâches réalisées par le groupe}
Chaque personne doit lire les documents fournis afin de comprendre les exigences et 
les instructions du projet. 

\noindent\rule{13cm}{0.4pt}

% Day 2 ---------------------------------------------------------------------
\section*{Day 2 - 17/12/2021}
Recherche de ressources utiles pour apprendre et comprendre ELF afin de mieux 
démarrer le projet et avoir une meilleure idée sur la manière de le lancer.

\subsection*{Tâches réalisées par le groupe}
\begin{itemize}
    \item Comprendre la structure d'un fichier ELF pour commencer la phase 1.
\end{itemize}

\noindent\rule{13cm}{0.4pt}

% Vacances ------------------------------------------------------------------
\section*{Vacances - 18/12/2021 à 02/01/2021}
Vacances de Nöel, aucun travail n'a été effectué.

\noindent\rule{13cm}{0.4pt}

% Day 11 --------------------------------------------------------------------
\section*{Day 11 - 26/12/2021}
Les fonctionnalités de readelf ont été testées pour comprendre les possibilités 
avant d'implémenter les fonctions pour la phase 1. Commencer à coder les fonctions 
de l'étape 1 (affichage de l'en tête).

\subsection*{Tâches réalisées par Kimberly Beauvais}
\begin{itemize}
    \item Création du code pour étape 1 : Affichage de l'en tête
    \begin{itemize}
        \item Création \textit{elf\_header.c}
        \item Création \textit{elf\_header.h}
    \end{itemize}
    \item Création de \textit{readelf.c} qui contient le programme principal 
    pour la phase 1
\end{itemize}

\noindent\rule{13cm}{0.4pt}

% Day 12 --------------------------------------------------------------------
\section*{Day 12 - 27/12/2021}
Le codage de l'étape 1 est terminé, on entre dans la phase de test. Commencer 
à coder les fonctions de l'étape 2 (affichage de la table des sections). 

\subsection*{Tâches réalisées par Kimberly Beauvais}
\begin{itemize}
    \item Création du code pour étape 2 : Affichage de la table des sections
    \begin{itemize}
        \item Création \textit{elf\_shdrs.c}
        \item Création \textit{elf\_shdrs.h}
    \end{itemize}
    \item Modifier \textit{readelf.c} pour ajouter la fonctionnalité 
    de l'affichage de la table des sections
\end{itemize}

\noindent\rule{13cm}{0.4pt}

% Vacances ------------------------------------------------------------------
\section*{Vacances - 28/12/2021 à 02/01/2022}
Vacances de Nouvel An, aucun travail n'a été effectué.

\noindent\rule{13cm}{0.4pt}

% Day 19 --------------------------------------------------------------------
\section*{Day 19 - 03/01/2021}
Commit de \textit{readelf} et \textit{elf\_header} dans le git (statut test). 
Commencer à coder les fonctions de l'étape 3 (affichage du contenu d'une 
section) et 4 (affichage de la table des symboles).

\subsection*{Tâches réalisées par Kimberly Beauvais}
\begin{itemize}
    \item Modification sur \textit{elf\_header} pour utiliser elf.h
    \item Commit \textit{readelf} et \textit{elf\_header} - test
    \item Continuation du code pour étape 2 \textit{elf\_shdrs}
\end{itemize}

\subsection*{Tâches réalisées par Xuan Li}
\begin{itemize}
    \item Création du code pour étape 3 : Affichage du contenu d'une section
    \begin{itemize}
        \item Création \textit{read\_section.c} en 64 bits
        \item Création \textit{read\_section.h}
    \end{itemize}
    \item Commit \textit{read\_section} - test
    \item Création du code pour étape 4 : Affichage de la table des symboles
    \begin{itemize}
        \item Création \textit{elf\_symbol\_table.c} en 64 bits
        \item Création \textit{elf\_symbol\_table.h}
    \end{itemize}
\end{itemize}

\noindent\rule{13cm}{0.4pt}

% Day 20 --------------------------------------------------------------------
\section*{Day 20 - 04/01/2021}
Commit de \textit{elf\_shdrs} et \textit{read\_symbol\_table} dans le git (statut
test). Edition de \textit{read\_symbol\_table} et \textit{read\_section} pour 
être en 32 bits. Commencer à coder les fonctions de l'étape 5 (Affichage des tables 
de réimplantation). Commit de \textit{elf\_reltab} (statut incomplète).


\subsection*{Tâches réalisées par Kimberly Beauvais}
\begin{itemize}
    \item Commit \textit{elf\_shdrs} - test
    \item Edition de \textit{readelf} pour ajouter les fonctionnalités des étapes
    3 et 4. 
    \item Clarification du code pour étape 1 et 2
\end{itemize}

\subsection*{Tâches réalisées par Xuan Li}
\begin{itemize}
    \item Commit \textit{read\_symbol\_table} - test
    \item Editition de \textit{read\_symbol\_table} et \textit{read\_section} 
    pour être en 32 bits
\end{itemize}

\subsection*{Tâches réalisées par Nathaniel Tobing}
\begin{itemize}
    \item Création du code pour étape 5 (Affichage des tables de réimplantation)
    \begin{itemize}
        \item Création \textit{elf\_reltab.c} 
        \item Création \textit{elf\_reltab.h}
    \end{itemize}
    \item Commit de \textit{elf\_reltab} - incomplète
\end{itemize}

\noindent\rule{13cm}{0.4pt}

% Day 21 --------------------------------------------------------------------
\section*{Day 21 - 05/01/2021}
Débogage et commit de \textit{elf\_symbol\_table} et \textit{read\_section} 
dans le git (statut incomplète). Continuation du code pour étape 5. Création 
des testes pour l'étape 1.

\subsection*{Tâches réalisées par Kimberly Beauvais}
\begin{itemize}
    \item Addition de print pour les OS/ABI de l'en tête du fichier
    \item Addition de print pour les flags de la table des sections
    \item Aide au débogage de l'étape 3 et 4 (transformation 64 en 32)
\end{itemize}

\subsection*{Tâches réalisées par Xuan Li}
\begin{itemize}
    \item Editition et débogage de \textit{read\_symbol\_table} et 
    \textit{read\_section} pour être en 32 bits
    \item Commit de \textit{elf\_symbol\_table} - incomplète
    \item Commit de \textit{read\_section} - incomplète
\end{itemize}

\subsection*{Tâches réalisées par Nathaniel Tobing}
\begin{itemize}
    \item Addition de plus de machines pour étape 1
    \item Edition de \textit{readelf} pour ajouter le fonctionnalité de l'étape 5
    \item Continuation du code pour étape 5 \textit{elf\_reltab}
\end{itemize}

\subsection*{Tâches réalisées par Emilien Maillard-Simon}
\begin{itemize}
    \item Création du code pour tester l'étape 1
    \begin{itemize}
        \item Création \textit{test\_header.sh} 
    \end{itemize} 
\end{itemize}

\subsection*{Tâches réalisées par Hugo Roger}
\begin{itemize}
    \item Création du programme shell pour automatiser les tests
    \item Explication des procédures aux membres du groupe ayant 
    eu des difficultés les ayant donnés des problèmes pour démarrer le projet
\end{itemize}

\subsection*{Tâches réalisées par Théo Lanneau}
\begin{itemize}
    \item Lecture de la documentation ELF pour une meilleure compréhension 
\end{itemize}

\noindent\rule{13cm}{0.4pt}

% Day 22 --------------------------------------------------------------------
\section*{Day 22 - 06/01/2021}
\textbf{Audit du code à 15h40.}

\noindent Termination du code pour la phase.
Débogage et commit de \textit{elf\_symbol\_table} et \textit{read\_section} 
dans le git (statut test). Commit de \textit{elf\_reltab} dans le git (statut 
test). Edition de \textit{readelf} pour ajouter des options. 

\subsection*{Tâches réalisées par Kimberly Beauvais}
\begin{itemize}
    \item Aide au débogage de l'étape 3 et 4
    \item Edition de \textit{readelf}
    \begin{itemize}
        \item Ajout des options
        \item Ajout d'un fonction pour obtenir toutes les données du fichier avant 
        de les processer selon les options
    \end{itemize}
\end{itemize}

\subsection*{Tâches réalisées par Xuan Li}
\begin{itemize}
    \item Débogage de l'étape 3 et 4
    \item Commit de \textit{elf\_symbol\_table} - test
    \item Commit de \textit{read\_section} - test
\end{itemize}

\subsection*{Tâches réalisées par Nathaniel Tobing}
\begin{itemize}
    \item Débogage de \textit{elf\_symbol\_table} pour l'utiliser en étape 5
    \item Commit de \textit{elf\_reltab} - test
\end{itemize}

\subsection*{Tâches réalisées par Hugo Roger}
\begin{itemize}
    \item Brainstorm d’idées pour pouvoir créer les jeux de tests

\end{itemize}

\noindent\rule{13cm}{0.4pt}

% Day 23 --------------------------------------------------------------------
\section*{Day 23 - 07/01/2021}
Dernières modifications pour la phase 1 sont ajoutées. 

\noindent \textbf{Début de la phase 2 - Réimplantation}

\noindent Commencer la recherche et discuter de la manière de mettre en œuvre 
la phase 2.


\subsection*{Tâches réalisées par Kimberly Beauvais}
\begin{itemize}
    \item Ajout des commentaires dans \textit{elf\_header}
    \item Ajout des commentaires dans \textit{elf\_shdrs}
    \item Création d'un \textit{README.md} pour la phase 1
\end{itemize}

\subsection*{Tâches réalisées par Xuan Li}
\begin{itemize}
    \item Ajout des commentaires dans \textit{elf\_symbol\_table}
    \item Ajout des commentaires dans \textit{read\_section}
\end{itemize}

\subsection*{Tâches réalisées par Nathaniel Tobing}
\begin{itemize}
    \item Modification des fonctions get\_filedata et process\_file dans \textit{readelf}
\end{itemize}

\subsection*{Tâches réalisées par Théo Lanneau}
\begin{itemize}
    \item Ecriture de nouveaux programmes ARM pour effectuer des tests sur la phase 1
\end{itemize}

\subsection*{Tâches réalisées par le groupe}
Recherche sur la phase 2 pour mieux comprendre le sujet.

\noindent\rule{13cm}{0.4pt}

% Day 24/25 --------------------------------------------------------------------
\section*{Week-end - 08/01/2021 à 09/01/2021}

Week-end, aucun travail n'a été effectué.

\noindent\rule{13cm}{0.4pt}

% Day 26 --------------------------------------------------------------------
\section*{Day 26 - 10/01/2021}
Début de la rédaction des documents à rendre.
Création du code pour les étapes 6 (Rénumeration des sections), 8 et 9 
(Réimplantation de type).


\subsection*{Tâches réalisées par Kimberly Beauvais}
\begin{itemize}
    \item Edition du \textit{README.md}
    \item Création d'un fichier LaTeX pour le compte rendu 
    \item Ajout des commentaires dans \textit{elf\_main}
\end{itemize}

\subsection*{Tâches réalisées par Xuan Li}
\begin{itemize}
    \item Création du code pour étape 8 (Réimplantation de type R\_ARM\_ABS*) et 
    9 (Réimplantation de type R\_ARM\_JUMP24 et R\_ARM\_CALL) 
    \begin{itemize}
        \item Création de \textit{reimplantation\_type.c}  
        \item Création de \textit{reimplantation\_type.h} 
    \end{itemize}
\end{itemize}

\subsection*{Tâches réalisées par Nathaniel Tobing}
\begin{itemize}
    \item Modification du programme principal de la phase 1 pour séparer les 
    fonctionnalités du programme (à utiliser dans phase 2) et le main 
    \begin{itemize}
        \item Création de \textit{elf\_main.c}  
        \item Création de \textit{elf\_main.h} 
    \end{itemize}
    \item Création du code pour étape 6 (Renumérotation des sections)
    \item Commit \textit{relocation.c} - incomplète
\end{itemize}

\subsection*{Tâches réalisées par Hugo Roger}
\begin{itemize}
    \item Réécriture des jeux de test pour la partie 1 (recommencement depuis zéro)
    \item Corrections pour chaque programme de la partie 1 pour pouvoir éviter les 
    conflits lors des tests et les adapter.
\end{itemize}

\noindent\rule{13cm}{0.4pt}

% Day 27 --------------------------------------------------------------------
\section*{Day 27 - 11/01/2021}

\subsection*{Tâches réalisées par Kimberly Beauvais}
\begin{itemize}
    \item Séparation des documents à rendre
    \begin{itemize}
        \item Création de \textit{description.tex} - Descriptive de la structure 
        de code développé
        \item Création de \textit{fonctionalites.tex} - Liste des 
        fonctionnalités implémentées et manquantes
        \item Création de \textit{bogues.tex} - Liste des éventuels bogues 
        \item Création de \textit{test.tex} - Liste et description des tests effectués
        \item Création de \textit{journal.tex} - Journal décrivant la progression
    \end{itemize}
    \item Continuation du CR pour la phase 1
\end{itemize}

\subsection*{Tâches réalisées par Xuan Li}
\begin{itemize}
    \item Commit \textit{reimplantation\_type} - incomplète
    \item Continuation du CR pour la phase 1
\end{itemize}

\subsection*{Tâches réalisées par Nathaniel Tobing}
\begin{itemize}
    \item Ajout de fonction copie fichier dans \textit{relocation} 
    \item Création du code pour étape 7 (Correction des symboles)
    \item Commit \textit{relocation.c} - test
\end{itemize}

\subsection*{Tâches réalisées par Hugo Roger}
\begin{itemize}
    \item Réécriture du programme d’automatisation des tests pour la partie 1 
    car des modifications ont été faites dans les programmes à exécuter
    \item Préparation pour la partie 2
\end{itemize}

\noindent\rule{13cm}{0.4pt}

% Day 28 --------------------------------------------------------------------
\section*{Day 28 - 12/01/2021}

\subsection*{Tâches réalisées par Kimberly Beauvais}
\begin{itemize}
    \item Continuation du CR
\end{itemize}

\subsection*{Tâches réalisées par Xuan Li}
\begin{itemize}
    \item Continuation du code et débogage pour les étapes 8 et 9
\end{itemize}

\subsection*{Tâches réalisées par Nathaniel Tobing}
\begin{itemize}
    \item Aide de débogage pour les étapes 8 et 9
\end{itemize}

\subsection*{Tâches réalisées par Théo Lanneau}
\begin{itemize}
    \item Commit de plusieurs programmes ARM. 
    \item Écriture d’un script bash pour comparer les Hexdumps de la fonction 
    readelf existante et de la fonction readelf que nous avons codé.
\end{itemize}

\subsection*{Tâches réalisées par Hugo Roger}
\begin{itemize}
    \item Conclusion de la partie 1 : tests
    \item Remplissage des parties bugs et tests dans le compte-rendu
    \item Création de tests pour la partie 2
\end{itemize}

\noindent\rule{13cm}{0.4pt}

% Day 29 --------------------------------------------------------------------
\section*{Day 29 - 13/01/2021}
Confirmation que le code fonctionne comme prévu et que les documents sont 
correctement formatés pour être remis.

\subsection*{Tâches réalisées par Kimberly Beauvais}
\begin{itemize}
    \item Continuation du CR
\end{itemize}

\subsection*{Tâches réalisées par Xuan Li}
\begin{itemize}
    \item Addition des commentaires pour étapes 8 et 9
\end{itemize}

\subsection*{Tâches réalisées par Nathaniel Tobing}
\begin{itemize}
    \item Addition des commentaires pour étapes 6 et 7
\end{itemize}

\subsection*{Tâches réalisées par Théo Lanneau}
\begin{itemize}
    \item Continuation des testes
\end{itemize}

\subsection*{Tâches réalisées par Hugo Roger}
\begin{itemize}
    \item Continuation des testes
\end{itemize}

\subsection*{Tâches réalisées par le groupe}
\begin{itemize}
    \item Vérification du code
    \item Vérification des documents
    \item Rendu du code à 12h
    \item Preparation pour le soutenance
\end{itemize}

\noindent\rule{13cm}{0.4pt}

% Check list ---------------------------------------------------------------
\section*{Checklist}
\begin{todolist}
    \item[$\boxtimes$] Phase 1 - Fusion (07/01/2022)
    \begin{todolist}
      \item[$\boxtimes$] Etape 1 - Affichage de l'en-tête  (03/01/2022)
      \item[$\boxtimes$] Etape 2 - Affichage de la table des sections (04/01/2022)
      \item[$\boxtimes$] Etape 3 - Affichage du contenu d'une section (06/01/2022)
      \item[$\boxtimes$] Etape 4 - Affichage de la table des symboles (06/01/2022)
      \item[$\boxtimes$] Etape 5 - Affichage des tables de réimplantation (06/01/2022)
    \end{todolist}
    \item Phase 2 - Réimplantation
    \begin{todolist}
        \item[$\boxtimes$] Etape 6 - Renumérotation des sections (11/01/2022)
        \item[$\boxtimes$] Etape 7 - Correction des symboles (11/01/2022)
        \item[$\boxtimes$] Etape 8 - Réimplantation de type R\_ARM\_ABS* (12/01/2022)
        \item[$\boxtimes$] Etape 9 - Réimplantation de type R\_ARM\_JUMP24 et 
        R\_ARM\_CALL (12/01/2022)
        \item Etape 10 - Interfaçage avec le simulateur ARM 
        \item Etape 11 - Production d'un fichier exécutable non relogeable
      \end{todolist}
  \end{todolist}


\end{document}