\documentclass[a4paper]{article} %type of paper

\usepackage[french]{babel} %paper language
\usepackage[utf8]{inputenc} %input language
\usepackage[T1]{fontenc} %fonts
\usepackage{amsmath} %insert advanced math
\usepackage{amssymb} %insert math symbols
\usepackage[hidelinks]{hyperref} %insert hyperlinks
\usepackage{graphicx} %insert images
\usepackage{listings} %insert code
\usepackage{soul} %underlines(ul)/barrer(st)
\usepackage{xcolor} %use new colors

% color for code
\definecolor{backColor}{HTML}{fdfdfd}
\definecolor{commentsColor}{HTML}{008000}
\definecolor{keywordsColor}{HTML}{1e7fe8}
\definecolor{numberColor}{HTML}{898989}
\definecolor{stringColor}{HTML}{bd4bb0}

% title
\title{PROG 5 Projet 2021-2022 \\
\large Fonctionnalités Implémentées et Manquantes}
% Author
\author{Kimberly Beauvais, Xuan Li, Nathaniel Tobing \\
Hugo Roger, Emilien Maillard-Simon, Théo Lanneau}


\lstdefinestyle{mystyle}{
    language=C,
    frame=single,
    backgroundcolor=\color{backColor},   
    commentstyle=\color{commentsColor},
    keywordstyle=\color{keywordsColor}\bfseries,
    numberstyle=\tiny\color{numberColor},
    stringstyle=\color{stringColor},
    basicstyle=\ttfamily\footnotesize,
    breakatwhitespace=false,         
    breaklines=true,                 
    captionpos=b,                    
    keepspaces=true,                 
    numbers=left,                    
    numbersep=10pt,                  
    showspaces=false,                
    showstringspaces=false,
    showtabs=false,                  
    tabsize=2
}

\lstset{style=mystyle}

\begin{document}

% title 
\maketitle
\begin{center}
\rule{\textwidth}{1pt}
\end{center}

\section*{Phase 1: ./readelf}

\subsection*{Implémentées :}
\begin{itemize}
    \item Affichage de l'en-tête
    \item Affichage de la table des sections et des détails relatifs à chaque 
    section
    \item Affichage du contenu d'une section
    \item Affichage de la table des symboles et des détails relatifs à chaque 
    symbole
    \item Affichage des tables de réimplantation et des détails relatifs à chaque 
    entrée
\end{itemize}


\subsection*{Manquantes :}
\begin{itemize}
    \item Possibilité de mettre plusieurs fichiers binaires à lire dans les arguments
\end{itemize}


\section*{Phase 2: ./relocation}

\subsection*{Implémentées :}
\begin{itemize}
    \item Renumérotation des sections
    \item Correction des symboles
    \item Réimplantations de type R\_ARM\_ABS* 
    \item Réimplantations de type R\_ARM\_JUMP24 et R\_ARM\_CALL
    \item Production d'un fichier ELF binaire modifiée
\end{itemize}


\subsection*{Manquantes :}
\begin{itemize}
    \item Possibilité de mettre plusieurs fichiers binaires à lire dans les arguments
    \item Interfaçage avec le simulateur ARM
    \item Production d'un vrai fichier exécutable non relogeable
\end{itemize}


\newpage

\end{document}